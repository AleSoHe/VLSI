%%%%%%%%%%%%%%%%%%%%%%%%%%%%%%%%%%%%%%%%%
% Engineering Calculation Paper
% LaTeX Template
% Version 1.0 (20/1/13)
%
% This template has been downloaded from:
% http://www.LaTeXTemplates.com
%
% Original author:
% Dmitry Volynkin (dim_voly@yahoo.com.au)
%
% License:
% CC BY-NC-SA 3.0 (http://creativecommons.org/licenses/by-nc-sa/3.0/)
%
% Modificaciones por Roberto Cerdas
%
% Si desea utilizar notas al margen, favor leer los comentarios en las líneas 32 y % 52. Si desea colocar un logo, favor leer comentario en línea 54. El comando     % \marginnote{texto} introduce notas al margen.  
%
%%%%%%%%%%%%%%%%%%%%%%%%%%%%%%%%%%%%%%%%%

%----------------------------------------------------------------------------------------
%	PACKAGES AND OTHER DOCUMENT CONFIGURATIONS
%----------------------------------------------------------------------------------------

\documentclass[12pt,a4paper]{article} % Use A4 paper with a 12pt font size - different paper sizes will require manual recalculation of page margins and border positions

\usepackage[spanish]{babel} % Utilizar reglas de idioma español
\usepackage[utf8]{inputenc} % Use UTF-8 encoding
\usepackage{marginnote} % Required for margin notes
\usepackage{wallpaper} % Required to set each page to have a background
\usepackage{lastpage} % Required to print the total number of pages
%\usepackage[left=1.3cm,right=4.6cm,top=1.8cm,bottom=4.0cm,marginparwidth=3.4cm]{geometry} % Comentar la línea abajo y descomentar esta para usar notas al margen
\usepackage[left=1.3cm,right=1.3cm,top=1.8cm,bottom=4.0cm]{geometry} % Adjust page margins
\usepackage{amsmath} % Required for equation customization
\usepackage{amssymb} % Required to include mathematical symbols
\usepackage{xcolor} % Required to specify colors by name
\usepackage[square, comma, sort&compress]{natbib} % Use the natbib reference package - read up on this to edit the reference style; if you want text (e.g. Smith et al., 2012) for the in-text references (instead of numbers), remove 'numbers' 

\usepackage{fancyhdr} % Required to customize headers
\setlength{\headheight}{80pt} % Increase the size of the header to accommodate meta-information
\pagestyle{fancy}\fancyhf{} % Use the custom header specified below
\renewcommand{\headrulewidth}{0pt} % Remove the default horizontal rule under the header

\setlength{\parindent}{0cm} % Remove paragraph indentation
\newcommand{\tab}{\hspace*{2em}} % Defines a new command for some horizontal space

\newcommand\BackgroundStructure{ % Command to specify the background of each page
\setlength{\unitlength}{1mm} % Set the unit length to millimeters

\setlength\fboxsep{0mm} % Adjusts the distance between the frameboxes and the borderlines
\setlength\fboxrule{0.5mm} % Increase the thickness of the border line
\put(10, 10){\fcolorbox{black}{white!10}{\framebox(192,247){}}} % Main content box
%\put(165, 10){\fcolorbox{black}{blue!10}{\framebox(37,247){}}} % Margin box: Descomentar para utilizar notas al margen.
\put(10, 262){\fcolorbox{black}{white!10}{\framebox(192, 25){}}} % Header box
%\put(143, 263){\includegraphics[height=23mm,keepaspectratio]{logo}} % Logo box - maximum height/width: 25x42. Descomentar esta línea para usar logo.
}

%----------------------------------------------------------------------------------------
%	HEADER INFORMATION
%----------------------------------------------------------------------------------------

\fancyhead[L]{\begin{tabular}{l r | l r} % The header is a table with 4 columns
\textbf{Proyecto} & Localizador Acústico & \textbf{Página} & \thepage/\pageref{LastPage} \\ % Project name and page count
\textbf{Trabajo} & Divisor & \textbf{Actualizado en:} & 08/12/2013 \\ % Job number and last updated date
\textbf{Versión} & 1 & \textbf{Revisado en:} & -/-/201- \\ % Version and reviewed date
\textbf{Diseñador} & Roberto Cerdas Robles & \textbf{Revisado por:} & Alfonso Chacón Rodríguez \\ % Designer and reviewer
\end{tabular}}

%----------------------------------------------------------------------------------------
